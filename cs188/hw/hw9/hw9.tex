%%% Please replace what is within each curly bracket with the correct information below. %%%
%%% The first field is already filled in for you.  %%%
\def\ClassName {CS188} % Your course 
\def\NameLast {Chen}  % Your last name
\def\NameFirst {Jianzhong}  % Your first name
\def\SID{23478230}  % Your SID
\def\Email{chenjianzhong@berkeley.edu} % Your pandagrader email
\def\Collaborators{None} % Any collaborators

\def\OneA{

\item \mcqb $A$ \item \mcqb $B$ \item \mcqs $C$ \item \mcqb $D$ \item \mcqb $E$ \item \mcqb $F$

}

\def\OneB{
\begin{itemize}[label=, itemsep=12pt, topsep=12pt]

\item \mcqb $-\infty$ \item \mcqb $1/2$ \item \mcqb $-2$ \item \mcqb $1$ \item \mcqb $-1$
\item \mcqb $2$ \item \mcqb $-1/2$ \item \mcqb $\infty$ \item \mcqs $0$
\end{itemize}
\end{multicols}

\begin{multicols}{2}
\begin{itemize}[label=, itemsep=12pt, topsep=12pt]
\item \mcqb $Can$ be determined but is not equal to any of the provided options
\item \mcqb $Cannot$ be determined from the information provided
\end{itemize}
}

\def\OneC{

\item \mcqs LL-Train would remain the same.
\item \mcqb LL-Train would go up.
\item \mcqb LL-Train would go down.
\item \mcqs LL-Test would remain the same.
\item \mcqb LL-Test would go up.
\item \mcqb LL-Test would go down.
}
\def\OneCExplanation{
Since LL-Train and LL-Test are under the limit of infinite data, and Laplace smoothing will only add only a small amount of samples, which is too few to change the value of LL-Train and LL-Test.
}

\def\OneD{

\begin{tabular}{llll}
Small amount of training data: &
\mcqb $A \qquad$ & \mcqs $B \qquad$ & \mcqb $C \qquad$ \\  \\ \\
Medium amount of training data: &
\mcqs $A$ & \mcqb $B$ & \mcqb$C$ \\ \\ \\
Large amount of training data: &
\mcqb $A$ & \mcqb $B$ & \mcqs$C$ \\ \\ \\
\end{tabular}
}

\def\OneDExplanation{
With fewer data, a simple Bayes' net can generalize pretty well, but a complex Bayes' net will overfit. However, for a large amount of data, a simple Bayes' net will underfit, but a complex Bayes' net can simulate the actually distribution.
}

\def\TwoA{
$p(A=+m|B=
+m,C=+m)=\frac{p(A=+m,B=
+m,C=+m)}{\sum_a{p(A=a,B=+m,C=+m)}}=\frac{p(A=+m)p(X=+m|\pi(X)=+m)p(X=+m|\pi(X)=+m)}{\sum_a{p(A=a)p(X=+m|\pi(X)=a)p(X=+m|\pi(X)=a)}}=\frac{0.5*0.9*0.9}{0.5*0.9*0.9+0.5*0.1*0.1}=\frac{81}{82}$
}

\def\TwoB{
$p(D=+m|A=+m)=\frac{0.9^2+0.1^2}{0.9^2*0.1^2+2*0.9*0.1}=\frac{41}{50}$
}

\def\TwoC{
$EU=p(D=+m|A=+m)*(-1)+(1-p(D=+m|A=+m))*1=-0.64$
}

\def\TwoD{
$VPI=1-0.64=0.36$
}

\def\TwoE{
Since D is independent from F given A, so the VPI of F given D=+m is 0.
}

\def\TwoF{
\begin{center}
\begin{tabular}{lllll}
\mcqs $A \phantom{----}$  & \mcqs $B \phantom{----}$  & \mcqb $C \phantom{----}$  & \mcqb $E \phantom{----}$   \\ [1em]
\mcqb $F \phantom{----}$  & \mcqb $G \phantom{----}$  & \mcqb $H \phantom{----}$  & \mcqb $I \phantom{----}$  & \mcqb $J \phantom{----}$   \\ [1em]
\mcqb $K \phantom{----}$  & \mcqb $L \phantom{----}$  & \mcqb $M \phantom{----}$  & \mcqb $N \phantom{----}$  & \mcqb $O \phantom{----}$   \\ [1em]
\end{tabular}
\end{center}
}
\def\TwoFExplanation{
Because D is dependent on B, and B is dependent on A.
}

\def\TwoG{
\begin{center}
\begin{tabular}{lllll}
\mcqs $A \phantom{----}$  & \mcqs $B \phantom{----}$  & \mcqb $C \phantom{----}$  & \mcqb $E \phantom{----}$   \\ [1em]
\mcqb $F \phantom{----}$  & \mcqb $G \phantom{----}$  & \mcqb $H \phantom{----}$  & \mcqb $I \phantom{----}$  & \mcqb $J \phantom{----}$   \\ [1em]
\mcqb $K \phantom{----}$  & \mcqb $L \phantom{----}$  & \mcqb $M \phantom{----}$  & \mcqb $N \phantom{----}$  & \mcqb $O \phantom{----}$   \\ [1em]
\end{tabular}
\end{center}
}

\def\TwoGExplanation{
Same reason as F
}
%%%%%%%%%%%%%%%%%%%%%% DO NOT CHANGE ANYTHING BELOW THIS LINE %%%%%%%%%%%%%%%%%%%%%%%%%
\def \showSolutions {} 
\documentclass[twoside]{article}

\usepackage{class}
%\usepackage{verbatim}
\usepackage{fancyhdr}
\usepackage{booktabs}
\usepackage{setspace}
\usepackage{amsmath,mathrsfs}
\usepackage{multicol}
\usepackage{multirow}
\usepackage{amssymb}
\usepackage{tikz}
\usetikzlibrary{matrix}
\usepackage{graphicx}
\usepackage{subfig}
\usepackage{array}
\usepackage{xcolor}
\usepackage{float}
\usepackage{enumitem}
\usepackage{mathcomp}
\usepackage{tabularx}
\usepackage{wasysym}
\usepackage{pbox}
\usetikzlibrary{bayesnet}

% Bubbles for multiple choice questions
\newcommand{\mcqbubble}{\bigcirc}
\newcommand{\mcqbubblefill}{\Large\newmoon}

% shorthand
\newcommand{\mcqb}{$\bigcirc$\ \ }
\newcommand{\mcqs}{\solution{\mcqb}{$\Large\newmoon$\ \ }}

% pruning
\newcommand{\prune}{\includegraphics[width=0.2in]{figures/red_x}}
% title is Written HW9
\title{Written HW9}
\begin{document}
\thispagestyle{empty}
\maketitle


\smallskip
\smallskip
\textbf{INSTRUCTIONS}

\begin{itemize}
\item \textbf{Due:} Monday, April 14th 2014 11:59 PM
\item \textbf{Policy:} Can be solved in groups (acknowledge collaborators) but must
be written up individually. However,
we strongly encourage you to first work alone for about 30 minutes total in order to simulate an exam environment.  Late homework
will not be accepted.
\item \textbf{Format:}
You must solve the questions on this handout (either through a pdf annotator, or by printing, then scanning; we recommend the latter to match exam setting). Alternatively, you can typeset a pdf on your own that has answers appearing in the same space (check edx/piazza for latex templating files and instructions).
\textbf{Make sure that your answers (typed or handwritten) are within the
dedicated regions for each question/part.  If you do not follow this format, we may deduct points.}

\item \textbf{How to submit:}  Go to www.pandagrader.com. Log in and click on the
class CS188 Spring 2014. Click
on the submission titled Written HW 9 and upload your pdf containing your answers. If this is your first time using
pandagrader, you will have to set your password before logging in the
first time.  To do so, click on "Forgot your password" on the login
page, and enter your email address on file with the registrar's office
(usually your @berkeley.edu email address). You will then receive an
email with a link to reset your password.

\end{itemize}


\begin{center}
\begin{tabular}{|r|c|}
\hline
\begin{minipage}{3cm}~\\Last Name~\\~\\\end{minipage} & \begin{minipage}[c][1cm][c]{8cm} ~ \NameLast \end{minipage}  \\
\hline
\begin{minipage}{3cm}~\\First Name~\\~\\\end{minipage} & \NameFirst \\
\hline
\begin{minipage}{3cm}~\\SID~\\~\\\end{minipage} & \SID \\
\hline
\begin{minipage}{3cm}~\\Email~\\~\\\end{minipage} & \Email \\
\hline
\begin{minipage}{3cm}~\\Collaborators~\\~\\\end{minipage} & \Collaborators \\
\hline

\end{tabular}
\end{center}



\vfill

\smallskip
\smallskip
\smallskip
\smallskip
\smallskip

\begin{center}
{\bf For staff use only}\\
\begin{Large}
\begin{tabular}{|r|r|r|}
\hline
Q. 1 & Q. 2 & Total\\
\hline

\quad/10 & \quad/20 & \qquad/30 \\
\hline
\end{tabular}\end{Large}
\end{center}


\newpage

\q{30}{CSPs}
\begin{question}[]{\bf Pacman's new house}

After years of struggling through mazes, Pacman has finally made peace with the ghosts, Blinky, Pinky, Inky, and Clyde, and invited them to live with him and Ms. Pacman. The move has forced Pacman to change the rooming assignments in his house, which has 6 rooms. He has decided to figure out the new assignments with a CSP in which the variables are Pacman \textbf{(P)}, Ms. Pacman \textbf{(M)}, Blinky \textbf{(B)}, Pinky \textbf{(K)}, Inky \textbf{(I)}, and Clyde \textbf{(C)}, the values are which room they will stay in, from 1-6, and the constraints are:
\begin{table}[h]
\centering
\begin{tabular}{ll}
i) No two agents can stay in the same room&\\
ii) \textbf{P} $>$ 3 &
vi) \textbf{B} is even\\
iii) \textbf{K} is less than \textbf{P}&
vii) \textbf{I} is not 1 or 6\\
iv) \textbf{M} is either 5 or 6&
viii) $\vert$\textbf{I}-\textbf{C}$\vert$ = 1\\
v) \textbf{P} $>$ \textbf{M}&
ix) $\vert$\textbf{P}-\textbf{B}$\vert$ = 2
\end{tabular}
\end{table}
\begin{subquestion}[1]{\bf Unary constraints}
On the grid below cross out the values from each domain that are eliminated by enforcing unary constraints.
\solution{
\begin{table}[h]
\centering
\begin{tabular}{ccccccc}
P & 1 & 2 & 3 & 4 & 5 & 6\\
B & 1 & 2 & 3 & 4 & 5 & 6\\
C & 1 & 2 & 3 & 4 & 5 & 6\\
K & 1 & 2 & 3 & 4 & 5 & 6\\
I & 1 & 2 & 3 & 4 & 5 & 6\\
M & 1 & 2 & 3 & 4 & 5 & 6\\
\end{tabular}
\end{table}
}{
\begin{table}[h]
\centering
\begin{tabular}{ccccccc}
\AnswerOneAi
\end{tabular}
\end{table}
}
\end{subquestion}
\begin{subquestion}[1]{\bf MRV}
According to the Minimum Remaining Value (MRV) heuristic, which variable should be assigned to first?\\\\
\AnswerOneAii

\end{subquestion}

\begin{subquestion}[3]{\bf Forward Checking}
For the purposes of decoupling this problem from your solution to the
previous problem, assume we choose to assign P first, and assign it the value 6. What are the resulting domains after enforcing unary constraints (from part i) and running forward checking for this assignment?
\solution{
\begin{table}[h]
\centering
\begin{tabular}{ccccccc}
P &   &   &   &  &   &  6\\
B & 1 & 2 & 3 & 4 & 5 & 6\\
C & 1 & 2 & 3 & 4 & 5 & 6\\
K & 1 & 2 & 3 & 4 & 5 & 6\\
I & 1 & 2 & 3 & 4 & 5 & 6\\
M & 1 & 2 & 3 & 4 & 5 & 6\\

\end{tabular}
\end{table}
}{
\begin{table}[h]
\begin{center}
\begin{tabular}{ccccccc}
\AnswerOneAiii
\end{tabular}
\end{center}
\end{table}

}
\end{subquestion}

\begin{subquestion}[3]{\bf Iterative Improvement}
Instead of running backtracking search, you decide to start over and run
iterative improvement with the min-conflicts heuristic for value selection. Starting with the following assignment:\\\\
P:6, B:4, C:3, K:2, I:1, M:5\\\\
First, for each variable write down how many constraints it violates in the table below.\\
Then, in the table on the right, for all variables that could be selected for assignment, put an x in any box that corresponds to a possible value that could be assigned to that variable according to min-conflicts \textbf{in the next iteration}. When marking next values a variable could take on, only mark values different from the current one.

\begin{center}
\begin{tabular}{cc}
\begin{tabular}{|c|c|}
\hline
\AnswerOneAiv
\end{tabular}
\end{tabular}
\end{center}
\end{subquestion}

\end{question}
\newpage
\begin{question}[]{\bf Variable ordering}\\
We say that a variable X is backtracked if, after a value has been assigned to X, the recursion returns at X without a solution, and a different value must be assigned to X.\\
For this problem, consider the following three algorithms:
\begin{enumerate}
\item
Run backtracking search with no filtering
\item 
Initially enforce arc consistency, then run backtracking search with no filtering
\item
Initially enforce arc consistency, then run backtracking search while enforcing arc consistency after each assignment\\
\end{enumerate}

\begin{subquestion}[6]{}\\
For each algorithm, circle all orderings of variable assignments that guarantee that no backtracking will be necessary when finding a solution to the CSP represented by the following constraint graph.\\\\
\begin{table}[h]
\begin{tabular}{cc}
\includegraphics[scale=.4]{figures/tree.png}
&
\begin{tabular}{c|c|c}
\AnswerOneBi
\end{tabular}
\end{tabular}
\end{table}\\

\end{subquestion}
\begin{subquestion}[6]{}\\
For each algorithm, circle all orderings of variable assignments that guarantee that no more than two variables will be backtracked when finding a solution to the CSP represented by the following constraint graph.\\\\

\begin{table}[h]
\centering
\begin{tabular}{cc}
\includegraphics[scale=.3]{figures/csp_graph.png}
&
\begin{tabular}{c|c|c}
\AnswerOneBii
\end{tabular}
\end{tabular}
\end{table}
\end{subquestion}
\end{question}
\newpage
\begin{question}[]{\bf All Satisfying Assignments}
Now consider a modified CSP in which we wish to find every possible satisfying assignment, rather than just one such assignment as in normal CSPs. In order to solve this new problem, consider a new algorithm which is the same as the normal backtracking search algorithm, except that when it sees a solution, instead of returning it, the solution gets added to a list, and the algorithm backtracks. Once there are no variables remaining to backtrack on, the algorithm returns the list of solutions it has found.\\\\
For each graph below, select whether or not using the MRV and/or LCV heuristics could affect the number of leaf nodes in the search tree in this new situation.


\begin{subquestion}[2]{}\\
\begin{tabular}{cl}
\multirow{1}{*}{\includegraphics[scale=0.5]{figures/disconnected.png} \hspace{1.4in}}
\AnswerOneCi

\end{tabular}\\
\end{subquestion}

\vspace{-.1in}
\begin{subquestion}[2]{}\\
\begin{tabular}{cl}


\multirow{1}{*}{\includegraphics[scale=0.5]{figures/chain.png} \hspace{0.260in}}
\AnswerOneCii

\end{tabular}\\
\end{subquestion}
\vspace{-.19in}
\begin{subquestion}[2]{}\\
\begin{tabular}{cl}

\multirow{1}{*}{\includegraphics[width=3in]{figures/tree.png}}\\\\
\AnswerOneCiii

\end{tabular}
\end{subquestion}
\vspace{-.19in}
\begin{subquestion}[2]{}\\
\begin{tabular}{cl}

\multirow{1}{*}{\includegraphics[width=3in]{figures/circle.png}} \\\\
\AnswerOneCiv

\end{tabular}
\end{subquestion}

\vspace{-.15in} 
\begin{subquestion}[2]{}\\
\begin{tabular}{cl}

\multirow{1}{*}{\includegraphics[width=3in]{figures/csp_graph.png}} \\\\
\AnswerOneCv

\end{tabular}
\end{subquestion}



\end{question}
\newpage
\begin{problem}[14]{Occupy Cal}

You are at Occupy Cal, and the leaders of the protest are deciding whether or not to march on California Hall. The decision is made centrally and communicated to the occupiers via the ``human microphone''; that is, those who hear the information repeat it so that it propagates outward from the center. This scenario is modeled by the following Bayes net:

\vspace{5mm}
\begin{figure}[htp]
\centering
\includegraphics[width=105mm]{figures/tree-labeled-tables-crop.pdf}
\end{figure}
\vspace{5mm}

Each random variable represents whether a given group of protestors hears instructions to march ($+m$) or not ($-m$). The decision is made at $A$, and both outcomes are equally likely. The protestors at each node relay what they hear to their two child nodes, but due to the noise, there is some chance that the information will be misheard. Each node except $A$ takes the same value as its parent with probability 0.9, and the opposite value with probability 0.1, as in the conditional probability tables shown.

\begin{question}[2] Compute the probability that node $A$ sent the order to march ($A = +m$) given that both $B$ and $C$ receive the order to march ($B=+m$, $C=+m$). \\
\solution{\vspace{0.5in}}{
   \fbox{\begin{minipage}[t][2.0cm][t]{18cm} 2a: \TwoA \end{minipage}}\\
}
\end{question}

\begin{question}[2] Compute the probability that $D$ receives the order $+m$ given that $A$ sent the order $+m$. \\
\solution{\vspace{0.5in}}{ 
   \fbox{\begin{minipage}[t][2.0cm][t]{18cm} 2b: \TwoB \end{minipage}}\\
}
\end{question}

You are at node $D$, and you know what orders have been heard at node $D$. Given your orders, you may either decide to march (\emph{march}) or stay put (\emph{stay}). (Note that these actions are distinct from the orders $+m$ or $-m$ that you hear and pass on. The variables in the Bayes net and their conditional distributions still behave exactly as above.) If you decide to take the action corresponding to the decision that was actually made at $A$ (not necessarily corresponding to your orders!), you receive a reward of $+1$, but if you take the opposite action, you receive a reward of $-1$.
\newpage
\begin{question}[2] Given that you have received the order $+m$, what is the expected utility of your optimal action? (Hint: your answer to part (b) may come in handy.) \\
\solution{\vspace{0.5in}}{
   \fbox{\begin{minipage}[t][1.2cm][t]{18cm} 2c: \TwoC \end{minipage}}\\
}
\end{question}


Now suppose that you can have your friends text you what orders they have received. (Hint: for the following two parts, you should not need to do much computation due to symmetry properties and intuition.)



\begin{question}[2] Compute the VPI of $A$ given that $D = +m$. \\
\solution{\vspace{0.5in}}{
   \fbox{\begin{minipage}[t][1.2cm][t]{18cm} 2d: \TwoD \end{minipage}}\\
}
\end{question}
\begin{question}[2] Compute the VPI of $F$ given that $D = +m$. \\
\solution{\vspace{0.5in}}{
   \fbox{\begin{minipage}[t][1.2cm][t]{18cm} 2e: \TwoE \end{minipage}}\\
}
\end{question}

For the following parts, you should select nodes in the accompanying diagrams that have the given properties. Use the quantities you have already computed and intuition to answer the following question parts; you should not need to do any computation.


\begin{question}[5] Select the nodes for which knowing the value of that node changes your belief about the decision made at $A$ given evidence at $D$ (i.e.~nodes $X$ such that $P(A|X,D) \neq P(A|D)$).

\vspace{-0.3cm}
\begin{center}
\includegraphics[width=120mm]{figures/tree-labeled-d-crop.pdf}
\end{center}
\TwoF
\solution{\vspace{0.5in}}{
   \fbox{\begin{minipage}[t][2.0cm][t]{18cm} 2f Explanation: \TwoFExplanation \end{minipage}}\\
}
\end{question}
\newpage
\begin{question}[5] Select the nodes which have nonzero VPI given evidence at $D$.

\vspace{5mm}
\begin{center}
\includegraphics[width=120mm]{figures/tree-labeled-d-crop.pdf}
\end{center}
\TwoG
\vspace{5mm}
\solution{\vspace{0.5in}}{
   \fbox{\begin{minipage}[t][2.0cm][t]{18cm} 2g Explanation: \TwoGExplanation \end{minipage}}\\
}
\end{question}
\end{problem}
\newpage


\end{document}