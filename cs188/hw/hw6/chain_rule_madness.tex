%!TEX root=fa13_midterm2.tex
\begin{problem}[]{Probability: Chain Rule Madness}

Note that the set of choices is the same for all of the following questions.
You may find it beneficial to read all the questions first, and to make a table on scratch paper.

%%% First part
\vspace{.2in}

\begin{question}[3]
Fill in the circles of \textbf{all} expressions that are equivalent to $\mathbf{P(A, B \mid C)}$,\\
\textbf{given no independence assumptions}:

\begin{multicols}{2}
\begin{itemize}[itemsep=6pt]
\OneA
\end{itemize}
\end{multicols}
\end{question}

\vspace{.2in}
\begin{question}[3]
Fill in the circles of \textbf{all} expressions that are equivalent to $\mathbf{P(A, B \mid C)}$,\\
\textbf{given that $\mathbf{A \indep B \mid C}$}:

\begin{multicols}{2}
\begin{itemize}[itemsep=6pt]
\OneB
\end{itemize}
\end{multicols}
\end{question}

%%% Second part
\vspace{.2in}

\begin{question}[3]
Fill in the circles of \textbf{all} expressions that are equivalent to $\mathbf{P(A \mid B, C)}$,\\
\textbf{given no independence assumptions}:

\begin{multicols}{2}
\begin{itemize}[itemsep=6pt]
\OneC
\end{itemize}
\end{multicols}
\end{question}

\vspace{.2in}

\begin{question}[3]
Fill in the circles of \textbf{all} expressions that are equivalent to $\mathbf{P(A \mid B, C)}$,\\
\textbf{given that $\mathbf{A \indep B \mid C}$}:

\begin{multicols}{2}
\begin{itemize}[itemsep=6pt]
\OneD
\end{itemize}
\end{multicols}
\end{question}


\end{problem}
