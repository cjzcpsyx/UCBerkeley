%%ES made this problem according to Pieter's suggestions below.

%%PA: how about making this a problem on variants of value iteration /
%%games, it could have 4 parts:
%%Part 1: grab Michael's question from "short questions" about costs
%%instead of rewards
%% Part 2: ask about what happens when the ghosts can choose from all
%% states s' that have non-zero probability according to T(s,a,s')
%% Part 3: ask if Pacman were to run Q-learning---and assume builds up
%% full table with one entry for each state action pair, and visits
%% every state-action pair infinitely often, and learning rate alpha
%% goes to zero the right way for Q-learning.   Will the learned
%% Q-values be equal to:  and then some options, including the minimax
%% option when able to search till the end of the game (which is the true answer)
%% Part 4: ask if Pacman were to run Q-learning but now with features
%% --- will it converge to the same answer as minimax (when able to
%% search till the end of the game and evaluate with same features at
%% end) --> false;
%%Part 5: maybe some other variant you can come up with, maybe related to your day-to-day change in policy (would require more thought though)

%% overall this will require some more thought; make sure to email
%% updates pdf to Dan once you have it (not clear that he has svn
%% access set up right now)

\begin{problem}[]{The Value of Games}

Pacman is the model of rationality and seeks to maximize his expected utility,\\
but that doesn't mean he never plays games.


\begin{question}[4] \textbf{A Costly Game.}
  Pacman is now stuck playing a new game with only costs and no payoff. Instead
  of maximizing expected utility $V(s)$, he has to minimize expected costs
  $J(s)$.  In place of a reward function, there is a cost function $C(s,a,s')$
  for transitions from $s$ to $s'$ by action $a$. We denote the discount
  factor by $\gamma \in (0,1)$. $J^*(s)$ is the expected cost incurred by the
  optimal policy. Which one of the following equations is satisfied by $J^*$?

\TwoA


\end{question}


\begin{question}[4] \textbf{It's a conspiracy again!}
  The ghosts have rigged the costly game so that once Pacman takes an action
  they can pick the outcome from all states $s' \in S'(s,a)$, the set of all
  $s'$ with non-zero probability according to $T(s,a,s')$. Choose the correct
  Bellman-style equation for Pacman against the adversarial ghosts.

\TwoB
\end{question}

\vspace{-.4in}

\end{problem}
